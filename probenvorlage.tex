\documentclass[a4paper,11pt, headsepline, footsepline]{scrreprt}
\usepackage[ngerman]{babel}
\usepackage{amsmath}
\usepackage[utf8]{inputenc}
\usepackage[T1]{fontenc}
\usepackage{lastpage}
\usepackage{tabularx}
\usepackage{scrpage2}
\usepackage[]{graphicx}
\usepackage{paralist}
\usepackage[babel,german=quotes]{csquotes}
\usepackage{geometry}
\usepackage[math]{iwona} %hübsche Schrift
\usepackage[decimalsymbol=comma]{siunitx} % Einheiten so: \SI{3.4}{kg} -> 3,4 kg
\usepackage{tikz}
\usepackage{fancybox}
\usepackage{calc}
\usepackage{picture}
\usepackage{ulem}
\usepackage{formular}
\usepackage{pifont}

\newcommand{\tick}{\ding{51}}
\newcommand{\cross}{\ding{55}}

\geometry{a4paper, portrait, inner=2.5cm, outer=2cm, top=2.5cm, bottom=2cm}

\newcounter{AufgabenCounter}
\setcounter{AufgabenCounter}{1}

\pagestyle{scrheadings}
\newcommand{\Thema}{Thema der Probe}
\newcommand{\Datum}{Datum}
\newcommand{\Fach}{Englisch}
\newcommand{\Probenart}{Kurzprobe}
\newcommand{\Seite}{Seite \thepage/\pageref{LastPage}}


% Nummer und Punkte selbst angeben
\newcommand{\AufgabeNrPkt}[2]{\vspace{5mm} \textbf{#1) \hfill (#2 Punkte)} }

% Nummer selbst angeben, aber keine Anzeige von Punkten 
\newcommand{\AufgabeNr}[1]{\vspace{5mm} \textbf{#1)} }

% Nummer automatisch und keine Anzeige von Punkten 
\newcommand{\Aufgabe}{%
\AufgabeNr{Aufgabe \theAufgabenCounter}%
\stepcounter{AufgabenCounter}}

% Nummer automatisch, Anzeige von Punkten 
\newcommand{\AufgabePkt}[1]{%
\AufgabeNrPkt{Aufgabe \theAufgabenCounter}{#1}%
\stepcounter{AufgabenCounter}}

\newcommand{\karos}[2]{
  \begin{tikzpicture}
    \draw[step=0.5cm,color=gray] (0,0) grid (#1 cm ,#2 cm);
  \end{tikzpicture}
}

\setlength{\parindent}{0pt} % Absatzeinrückung von Links

\begin{document}
\lohead{Name:\\}
\rohead{\Fach \\ \Probenart, \Datum}

\cfoot{} 
% leere Fußzeile, damit keine Seitenzahl in der Fußzeile erscheint. 
% Wir haben ja eine in der Kopfzeile
\lofoot{z.B Name der Schule}
\rofoot{\Seite}

\cohead{\large{\Thema}}

\newFRMfield{line}{10cm}


\AufgabePkt{2}\\
Schreibe in der Höflichkeitsform.
\begin{enumerate}[a)]
\item Ich hätte gerne einen Apfel.\\\\
\useFRMfield{line}
\item Wir hätten gerne Wasser.\\\\
\useFRMfield{line}
\end{enumerate}

\AufgabePkt{2}\\
\textit{simple present}\\
\ovalbox{\textit{Hinweis:}
\textit{\tick =  simple present}
\textit{\cross = simple present verneint}}
\begin{enumerate}[a)]
\newFRMfield{verb_watch}{30mm}[watch]
\newFRMfield{verb_visit}{40mm}[visit]
\newFRMfield{verb_go}{30mm}[go]
\newFRMfield{verb_like}{40mm}[like]
\item \tick \ Tom \useFRMfield{verb_watch} the football match.
\item \cross \ Sally \useFRMfield{verb_visit} her aunt Claire.
\item \tick \ Peter and Simon \useFRMfield{verb_go} to school.
\item \cross \ We \useFRMfield{verb_like} school.
\end{enumerate}

\AufgabePkt{2}\\
\textit{simple past}\\
\ovalbox{\textit{Hinweis:}
\textit{\tick = simple past}
\textit{\cross = simple past verneint}}
\begin{enumerate}[a)]
\newFRMfield{verb_walk}{30mm}[walk]
\newFRMfield{verb_bring}{40mm}[bring]
\newFRMfield{verb_answer}{40mm}[answer]
\newFRMfield{verb_plan}{30mm}[plan]
\item \cross \ Timothy \useFRMfield{verb_walk} to the beach.
\item \tick \ Alex \useFRMfield{verb_bring} some cake.
\item \tick \ Monica \useFRMfield{verb_answer} the phone.
\item \cross \ The Smiths \useFRMfield{verb_plan} their trip to Germany.
\end{enumerate}

\AufgabePkt{2}\\
\textit{future}\\
\ovalbox{\textit{Hinweis:}
\textit{\tick = future}
\textit{\cross = future verneint}}
\begin{enumerate}[a)]
\newFRMfield{verb_eat}{40mm}[eat]
\newFRMfield{verb_drive}{40mm}[drive]
\newFRMfield{verb_write}{40mm}[write]
\newFRMfield{verb_talk}{40mm}[talk]
\item \tick \ The Smiths \useFRMfield{verb_eat} some burgers.
\item \tick \ Anthony \useFRMfield{verb_drive} home soon.
\item \cross \ The student \useFRMfield{verb_write} a test.
\item \cross \ I \useFRMfield{verb_talk} to you now.
\end{enumerate}

\newpage
\AufgabePkt{5}\\\\
\huge
\begin{tabular}{|l|r|}
	\hline
	English & German \\
	\hline
	elk & \\
	\hline
	 & Unabhängigkeit \\
	\hline
	 & Taschenlampe \\
	\hline
	shining & \\
	\hline
	 & (Menschen-)Menge \\
	\hline
	equipment & \\
	\hline
	scissors & \\
	\hline
	 & Abfall \\
	\hline
	 & Seife \\
	\hline
	buck \textit{(informal, AE)} & \\
	\hline
\end{tabular}

{\bigskip {\large \textsl{Viel Erfolg!}}}

\end{document}
