\documentclass[a4paper, parskip]{scrartcl}

\usepackage[ngerman]{babel}
%\usepackage[T1]{fontenc}
%\usepackage[utf8]{inputenc}
\usepackage{graphicx}
\usepackage{fontspec}
\usepackage{titlesec}
\setmainfont{Century Gothic}
\setsansfont{Century Gothic}

\title{Nutzungsordnung der IT-Ausstattung und des Internets an der
Mittelschule Markt Indersdorf für Lehrkräfte}

\date{}

\begin{document}

\begin{figure}[h]
	\flushright
	\includegraphics[width=7cm]{logo_briefkopf}
	\maketitle
\end{figure}

\section{Allgemeines}
Die Mittelschule Markt Indersdorf gibt sich für die Benutzung von schulischen
Computereinrichtungen mit Internetzugang die folgende Nutzungsordnung. Diese
Nutzungsordnung gilt für die Nutzung von Computern und des Internets durch
Lehrkräfte im Rahmen des Unterrichts, der Gremienarbeit, außerhalb des
Unterrichts zu unterrichtlichen Zwecken sowie zu privaten Zwecken. Auf eine
rechnergestützte Schulverwaltung findet die Nutzungsordnung keine Anwendung.

\section{Regeln für jede Nutzung}
\subsection{Schutz der Geräte}
Die Bedienung der Hard- und Software hat entsprechend den vorgegebenen
Instruktionen zu erfolgen. Störungen oder Schäden sind sofort dem
Systembetreuer zu melden. Wer schuldhaft Schäden verursacht, hat diese zu
ersetzen. Elektronische Geräte sind durch Schmutz und Flüssigkeiten besonders
gefährdet; deshalb sind während der Nutzung der Schulcomputer Essen und Trinken
verboten.
\subsection{Anmeldung an den Computern}
Nach Beendigung der Nutzung hat sich die Lehrkraft am Computer bzw. beim benutzten
Dienst abzumelden. Für Handlungen im Rahmen der schulischen Internetnutzung ist
die jeweilige Lehrkraft verantwortlich. Das Passwort muss vertraulich behandelt
werden. Das Arbeiten unter einem fremden Passwort ist verboten. Wer vermutet,
dass sein Passwort anderen Personen bekannt geworden ist, ist verpflichtet,
dieses zu ändern.
\subsection{Eingriffe in die Hard- und Softwareinstallation}
Veränderungen der Installation und Konfiguration der Arbeitsstationen und des
Netzwerks sowie Manipulationen an der Hardwareausstattung sind grundsätzlich
untersagt. Dies gilt nicht, wenn Veränderungen auf Anordnung des
Systembetreuers durchgeführt werden oder wenn temporäre Veränderungen im Rahmen
des Unterrichts explizit vorgesehen sind. Fremdgeräte (beispielsweise
Peripheriegeräte wie externe Datenspeicher oder persönliche Notebooks) dürfen
grundsätzlich nur mit Zustimmung des Systembetreuers an Computer oder an das
Netzwerk angeschlossen werden. Unnötiges Datenaufkommen durch Laden und
Versenden großer Dateien (etwa Filme) aus dem Internet ist zu vermeiden. Sollte
ein Nutzer unberechtigt größere Datenmengen in seinem Arbeitsbereich ablegen,
ist die Schule berechtigt, diese Daten zu löschen.
\subsection{Verbotene Nutzungen}
Die gesetzlichen Bestimmungen - insbesondere des Strafrechts, des Urheberrechts
und des Jugendschutzrechts - sind zu beachten. Es ist verboten,
pornographische, gewaltverherrlichende oder rassistische Inhalte aufzurufen
oder zu versenden. Werden solche Inhalte versehentlich aufgerufen, ist die
Anwendung zu schließen. Verboten ist beispielsweise auch die Nutzung von
Online-Tauschbörsen.
\subsection{Protokollierung des Datenverkehrs}
Die Schule ist berechtigt, den Datenverkehr während der Internetnutzung im
Unterricht und außerhalb des Unterrichts zu unterrichtlichen Zwecken zu
speichern und zu kontrollieren. Diese Daten werden in der Regel nach einem
Monat, spätestens jedoch nach einem halben Jahr gelöscht. Dies gilt nicht,
wenn Tatsachen den Verdacht eines schwerwiegenden Missbrauches der schulischen
Computer begründen. Der Schulleiter oder von ihm beauftragte Personen werden
von ihren Einsichtsrechten nur stichprobenartig oder im Einzelfall in Fällen
des Verdachts von Missbrauch Gebrauch machen.
\subsection{Nutzung von Informationen aus dem Internet}
Die Nutzung des Internets im Unterricht und außerhalb des Unterrichts zu
unterrichtlichen Zwecken ist zulässig. Als schulisch ist ein elektronischer
Informationsaustausch anzusehen, der unter Berücksichtigung seines Inhalts
und des Adressatenkreises mit der schulischen Arbeit im Zusammenhang steht.
Das Herunterladen von Anwendungen ist nur mit Einwilligung der Schule zulässig.
Die Schule ist nicht für den Inhalt der über ihren Zugang abrufbaren
Angebote Dritter im Internet verantwortlich. Im Namen der Schule dürfen weder
Vertragsverhältnisse eingegangen noch ohne Erlaubnis kostenpflichtige Dienste
im Internet benutzt werden. Beim Herunterladen wie bei der Weiterverarbeitung
von Daten aus dem Internet sind insbesondere Urheber- oder Nutzungsrechte zu
beachten.
\subsection{Verbreiten von Informationen im Internet}
Werden Informationen im bzw. über das Internet verbreitet, geschieht das unter
Beachtung der allgemein anerkannten Umgangsformen. Die Veröffentlichung von
Internetseiten der Schule bedarf der Genehmigung durch die Schulleitung. Für
fremde Inhalte ist insbesondere das Urheberrecht zu beachten. So dürfen
beispielsweise digitalisierte Texte, Bilder und andere Materialien nur mit
Zustimmung des Rechteinhabers auf eigenen Internetseiten verwandt oder über das
Internet verbreitet werden. Der Urheber ist zu nennen, wenn dieser es
wünscht. Das Recht am eigenen Bild ist zu beachten.
Daten von Schülerinnen und Schülern sowie Erziehungsberechtigten dürfen auf der
Internetseite der Schule nur veröffentlicht werden, wenn die Betroffenen
wirksam eingewilligt haben. Bei Minderjährigen bis zur Vollendung des 14.
Lebensjahres ist dabei die Einwilligung der Erziehungsberechtigten, bei
Minderjährigen ab der Vollendung des 14. Lebensjahres deren Einwilligung und
die Einwilligung der Erziehungsberechtigten erforderlich. Die Einwilligung kann
widerrufen werden. In diesem Fall sind die Daten zu löschen. Für den Widerruf
der Einwilligung muss kein Grund angegeben werden.
Hinsichtlich der Veröffentlichung von Daten von Lehrkräften oder der
Schulleitung auf der Internetseite der Schule gilt Folgendes:
Von der Schulleitung oder von Lehrkräften, die an der Schule eine Funktion mit
Außenwirkung wahrnehmen, dürfen ohne deren Einwilligung lediglich der Name,
Namensbestandteile, Vorname(n), Funktion, Amtsbezeichnung, Lehrbefähigung,
dienstliche Anschrift, dienstliche Telefonnummer und die dienstliche
E-Mail-Adresse angegeben werden. Andere Daten dieser Personen (wie etwa Fotos,
Sprechzeiten), dürfen nur veröffentlicht werden, wenn die Betroffenen in die
Veröffentlichung auf den Internetseiten der Schule wirksam eingewilligt
haben.
Daten von Lehrkräften (beispielsweise Sprechzeiten), die an der Schule keine
Funktion mit Außenwirkung wahrnehmen, dürfen auf den Internetseiten der
Schule nur veröffentlicht werden, wenn die Betroffenen wirksam eingewilligt
haben. Die Einwilligung kann widerrufen werden. In diesem Fall sind die Daten
zu löschen. Für den Widerruf der Einwilligung muss kein Grund angegeben werden.
Vertretungspläne dürfen ohne schriftliche Zustimmung aller betroffenen
Lehrkräfte nicht auf den Internetseiten der Schule veröffentlicht werden. Da
die Zustimmung in jedem Einzelfall eingeholt werden müsste und dies in der
Praxis kaum realisierbar ist, ist aus Datenschutzgründen auf eine
Veröffentlichung der Vertretungspläne auf der Internetseite der Schule zu verzichten.
Indem lediglich der geänderte Zeitpunkt des Unterrichtsbeginns bzw. des
Unterrichtsendes bzw. die Änderung des Unterrichtsfachs im Internet mitgeteilt
wird, kann eine ausreichende Information auch in nicht-personenbezogener Weise
erfolgen. In diesem Fall ist keine Zustimmung der betroffenen Lehrkräfte
notwendig.
Wegen der besonderen Öffentlichkeitswirksamkeit des Internets sind die
Betroffenen in jedem Fall - auch beim Vorliegen einer Einwilligung - vor der
Veröffentlichung in geeigneter Weise zu informieren.

\section{Ergänzende Regeln für die Nutzung der IT-Ausstattung und des
Internets zu privaten Zwecken}
\subsection{Protokollierung des Datenverkehrs}
Bei der Nutzung der IT-Ausstattung und des Internets zu privaten Zwecken ist
eine inhaltliche Kontrolle und Protokollierung der Internetaktivitäten durch
die Schule ohne vorherige Einwilligung der Lehrkraft unzulässig, da die Schule
in diesem Fall als Anbieter einer Dienstleistung nach dem
Telekommunikationsgesetz (TKG) anzusehen ist und die anfallenden Nutzungsdaten
(beispielsweise Webseitenaufruf) nur zu Abrechnungszwecken verwenden, aber
nicht inhaltlich prüfen darf (hierzu § 88 Abs. 3 TKG). Nur nach vorheriger
Einwilligung der Lehrkraft können die Internetaktivitäten inhaltlich
kontrolliert und protokolliert werden. Daher ist die vorherige Einwilligung
der Lehrkraft Voraussetzung für eine Zulassung zur Nutzung der IT-Ausstattung
und des Internets zu privaten Zwecken.
Die Lehrkraft kann die Einwilligung jederzeit widerrufen. Im Falle des
Widerrufs ist die Nutzung der IT-Ausstattung und des Internets zu privaten
Zwecken nicht mehr gestattet.

\section{Zuständigkeiten}
\subsection{Verantwortlichkeit der Schulleitung}
Die Schulleitung ist dafür verantwortlich, eine Nutzungsordnung entsprechend
dem in der jeweiligen Schulordnung vorgesehen Verfahren aufzustellen. Sie hat
den Systembetreuer, den Webmaster, die Lehrkräfte wie auch aufsichtführende
Personen über die Geltung der Nutzungsordnung zu informieren. Insbesondere hat
sie dafür zu sorgen, dass die Nutzungsordnung in den Räumen der Schule, in
denen eine Nutzung des Internets möglich ist, angebracht wird. Folgerichtig ist
die Nutzungsordnung auch an dem Ort, an dem Bekanntmachungen der Schule
üblicherweise erfolgen, anzubringen. Die Schulleitung hat die Einhaltung der
Nutzungsordnung stichprobenartig zu überprüfen. Die Schulleitung ist ferner
dafür verantwortlich, dass bei einer Nutzung des Internets im Unterricht und
außerhalb des Unterrichts zu unterrichtlichen Zwecken eine ausreichende
Aufsicht sichergestellt ist. Sie hat diesbezüglich organisatorische Maßnahmen
zu treffen. Des Weiteren ist die Schulleitung dafür verantwortlich, über den
Einsatz technischer Vorkehrungen zu entscheiden.
Die Schulleitung trägt die Verantwortung für die Schulwebsite.
\subsection{Verantwortlichkeit des Systembetreuers}
Der Systembetreuer hat in Abstimmung mit dem Lehrerkollegium, der Schulleitung
und dem Sachaufwandsträger über die Gestaltung und Nutzung der schulischen
IT-Infrastruktur zu entscheiden und regelt dazu die Details und überprüft die
Umsetzung:

\begin{itemize}
	\item{Nutzung der schulischen IT-Infrastruktur (Zugang mit oder ohne
		individuelle Authentifizierung, klassenbezogener Zugang,
	platzbezogener Zugang),}
	\item{Nutzung persönlicher mobiler Endgeräte und Datenspeicher
		(beispielsweise USB-Sticks) im Schulnetz,}
	\item{Technische Vorkehrungen zur Absicherung des Internetzugangs (wie
		etwa Firewallregeln, Webfilter, Protokollierung).}
\end{itemize}

\subsection{Verantwortlichkeit des Webmasters}
Der Webmaster hat in Abstimmung mit dem Lehrerkollegium, der Schulleitung und
gegebenenfalls weiteren Vertretern der Schulgemeinschaft über die Gestaltung
und den Inhalt des schulischen Webauftritts zu entscheiden. Er regelt dazu die
Details und überprüft die Umsetzung. Zu seinen Aufgaben gehören:

\begin{itemize}
	\item{Auswahl eines geeigneten Webhosters in Abstimmung mit dem
		Sachaufwandsträger,}
	\item{Vergabe von Berechtigungen zur Veröffentlichung auf der
		schulischen Website,}
	\item{überprüfung der datenschutzrechtlichen Vorgaben, insbesondere bei
		der Veröffentlichung persönlicher Daten und Fotos,}
	\item{Regelmäßige überprüfung der Inhalte der schulischen Webseiten.}
\end{itemize}

\subsection{Verantwortlichkeit der Lehrkräfte}
Die Lehrkräfte sind für die Beaufsichtigung der Schülerinnen und Schüler bei
der Nutzung der IT-Ausstattung und des Internets im Unterricht und außerhalb
des Unterrichts zu unterrichtlichen Zwecken verantwortlich.
\subsection{Verantwortlichkeit der aufsichtführenden Personen}
Die aufsichtführenden Personen haben auf die Einhaltung der Nutzungsordnungen
durch die Schülerinnen und Schüler hinzuwirken.
\subsection{Verantwortlichkeit der Nutzerinnen und Nutzer}
Die Schülerinnen und Schüler haben das Internet verantwortungsbewusst zu
nutzen. Sie dürfen bei der Nutzung der IT-Ausstattung und des Internets nicht
gegen gesetzliche Vorschriften verstoßen. Sie haben die Regelungen der
Nutzungsordnung einzuhalten.

\section{Schlussvorschriften}
Diese Nutzungsordnung ist Bestandteil der gültigen Hausordnung und tritt am
Tage nach ihrer Bekanntgabe durch Aushang in der Schule in Kraft. Einmal zu
jedem Schuljahresbeginn findet eine Nutzerbelehrung statt, die im Klassenbuch
protokolliert wird. Nutzer, die unbefugt Software von den Arbeitsstationen oder
aus dem Netzwerk kopieren oder verbotene Inhalte nutzen, können strafrechlich
sowie zivilrechtlich belangt werden. Zuwiderhandlungen gegen diese
Nutzungsordnung können neben dem Entzug der Nutzungsberechtigung
schulordnungsrechtliche Maßnahmen zur Folge haben.

\end{document}
