\documentclass[a4paper, parskip]{scrartcl}

\usepackage[ngerman]{babel}
\usepackage[T1]{fontenc}
\usepackage[utf8]{inputenc}
\usepackage{graphicx}

\title{Nutzungsordnung der IT-Ausstattung und des Internets an der
Mittelschule Markt Indersdorf}

\date{}

\begin{document}

\begin{figure}[h]
	\flushright
	\includegraphics[width=7cm]{logo_briefkopf}
	\maketitle
\end{figure}

\section{Allgemeines}
Die IT-Ausstattung der Schule und das Internet können als Lehr- und Lernmittel
genutzt werden. Dadurch ergeben sich vielfältige Möglichkeiten, pädagogisch
wertvolle Informationen abzurufen. Gleichzeitig besteht jedoch die Gefahr, dass
Schülerinnen und Schüler Zugriff auf Inhalte erlangen, die ihnen nicht zur
Verfügung stehen sollten. Weiterhin ermöglicht das Internet den Schülerinnen
und Schülern, eigene Inhalte weltweit zu verbreiten.  Die Mittelschule Markt
Indersdorf gibt sich deshalb für die Benutzung von schulischen
Computereinrichtungen mit Internetzugang die folgende Nutzungsordnung. Diese
gilt für die Nutzung von Computern und des Internets durch Schülerinnen und
Schüler im Rahmen des Unterrichts, der Gremienarbeit sowie außerhalb des
Unterrichts zu unterrichtlichen Zwecken. Auf eine rechnergestützte
Schulverwaltung findet die Nutzungsordnung keine Anwendung.

\section{Regeln für die Benutzung}
\subsection{Schutz der Geräte}
Die Bedienung der Hard- und Software hat entsprechend den vorhandenen
Instruktionen zu erfolgen. Störungen oder Schäden sind sofort der
aufsichtführenden Person zu melden. Wer schuldhaft Schäden verursacht, hat
diese zu ersetzen. Bei Schülerinnen und Schülern, die das 18. Lebensjahr noch
nicht vollendet haben, hängt die deliktische Verantwortlichkeit von der für die
Erkenntnis erforderlichen Einsicht ab (§ 823 Abs. 3 Bürgerliches Gesetzbuch –
BGB). Elektronische Geräte sind durch Schmutz und Flüssigkeiten besonders
gefährdet; deshalb sind während der Nutzung der Schulcomputer Essen und Trinken
verboten.
\subsection{Anmeldung an den Computern}
Die Nutzung mancher Computer bedarf einer Authentifizierung. Ebenso ist zur
Nutzung bestimmter Dienste im Internet (z.B. Lernplattformen) eine Anmeldung
mit Benutzernamen und Passwort erforderlich. Diese werden den Schülern zu
gegebenem Anlass mitgeteilt. Nach Beendigung der Nutzung haben sich die
Schülerinnen und Schüler am Computer und von den Diensten wieder abzumelden.
Für Handlungen im Rahmen der schulischen Internetnutzung sind die jeweiligen
Schülerinnen und Schüler verantwortlich. Das Passwort muss vertraulich
behandelt werden. Das Arbeiten unter einem fremden Passwort ist verboten. Wer
vermutet, dass sein Passwort anderen Personen bekannt geworden ist, ist
verpflichtet, dies dem Systembetreuer der Schule mitzuteilen.
\subsection{Eingriffe in die Hard- und Softwareinstallation}
Veränderungen der Installation und Konfiguration der Computer und des Netzwerks
sowie Manipulationen an der Hardwareausstattung sind grundsätzlich untersagt.
Dies gilt nicht, wenn Veränderungen auf Anordnung des Systembetreuers
durchgeführt werden oder wenn temporäre Veränderungen im Rahmen des Unterrichts
explizit vorgesehen sind. Fremdgeräte (beispielsweise Peripheriegeräte wie
Datenspeicher oder persönliche Notebooks) dürfen nur mit Zustimmung des
Systembetreuers, einer Lehrkraft oder aufsichtführenden Person am Computer oder
an das Netzwerk angeschlossen werden. Unnötiges Datenaufkommen durch Laden und
Versenden großer Dateien (etwa Filme) aus dem Internet ist zu vermeiden. Sollte
ein Nutzer unberechtigt größere Datenmengen in seinem Arbeitsbereich ablegen,
ist die Schule berechtigt, diese Daten zu löschen.
\subsection{Verbotene Nutzungen}
Die gesetzlichen Bestimmungen – insbesondere des Strafrechts, des Urheberrechts
und des Jugendschutzrechts – sind zu beachten. Es ist verboten,
pornographische, gewaltverherrlichende oder rassistische Inhalte abzurufen oder
zu versenden. Werden solche Inhalte versehentlich aufgerufen, ist die Anwendung
zu schließen und der Aufsichtsperson Mitteilung zu machen. Verboten ist
beispielsweise auch die Nutzung von Online-Tauschbörsen.
\subsection{Protokollierung des Datenverkehrs}
Die Schule ist in Wahrnehmung ihrer Aufsichtspflicht berechtigt, den
Datenverkehr zu speichern und zu kontrollieren. Diese Daten werden in der Regel
nach einem Monat, spätestens jedoch nach einem halben Jahr gelöscht. Dies gilt
nicht, wenn Tatsachen den Verdacht eines schwerwiegenden Missbrauches der
schulischen Computer begründen. In diesem Fall sind die personenbezogenen Daten
bis zum Abschluss der Prüfungen und Nachforschungen in diesem Zusammenhang zu
speichern. Die Schulleiterin/Der Schulleiter oder von ihm/ihr beauftragte
Personen werden von ihren Einsichtsrechten nur stichprobenartig oder im
Einzelfall in Fällen des Verdachts von Missbrauch Gebrauch machen.
\subsection{Nutzung von Informationen aus dem Internet}
Die Nutzung der IT-Ausstattung und des Internets ist nur im Unterricht und
außerhalb des Unterrichts zu unterrichtlichen Zwecken zulässig. Die Nutzung der
IT-Ausstattung und des Internets zu privaten Zwecken ist nicht gestattet. Als
schulisch ist ein elektronischer Informationsaustausch anzusehen, der unter
Berücksichtigung seines Inhalts und des Adressatenkreises mit der schulischen
Arbeit im Zusammenhang steht. Das Herunterladen von Anwendungen ist nur mit
Einwilligung der Schule zulässig. Die Schule ist nicht für den Inhalt der über
ihren Zugang abrufbaren Angebote Dritter im Internet verantwortlich. Im Namen
der Schule dürfen weder Vertragsverhältnisse eingegangen noch ohne Erlaubnis
kostenpflichtige Dienste im Internet benutzt werden. Beim Herunterladen wie bei
der Weiterverarbeitung von Daten aus dem Internet sind insbesondere Urheber-
oder Nutzungsrechte zu beachten.
\subsection{Verbreiten von Informationen im Internet}
Werden Informationen im bzw. über das Internet verbreitet, geschieht das unter
Beachtung der allgemein anerkannten Umgangsformen. Die Veröffentlichung von
Internetseiten der Schule bedarf der Genehmigung durch die Schulleitung. Für
fremde Inhalte ist insbesondere das Urheberrecht zu beachten. So dürfen
beispielsweise digitalisierte Texte, Bilder und andere Materialien nur mit
Zustimmung des Rechteinhabers auf eigenen Internetseiten verwandt oder über das
Internet verbreitet werden. Der Urheber ist zu nennen, wenn dieser es wünscht.
Das Recht am eigenen Bild ist zu beachten. Daten von Schülerinnen und Schülern
sowie Erziehungsberechtigten dürfen auf den Internetseiten der Schule nur
veröffentlicht werden, wenn die Betroffenen wirksam eingewilligt haben. Bei
Minderjährigen bis zur Vollendung des 14. Lebensjahres ist dabei die
Einwilligung der Erziehungsberechtigten, bei Minderjährigen ab 14 Lebensjahren
deren Einwilligung und die Einwilligung der Erziehungsberechtigten
erforderlich. Die Einwilligung kann widerrufen werden. In diesem Fall sind die
Daten zu löschen. Für den Widerruf der Einwilligung muss kein Grund angegeben
werden. Die Schülerinnen und Schüler werden auf die Gefahren hingewiesen, die
mit der Verbreitung persönlicher Daten im Internet einhergehen. Weiterhin wird
auf einen verantwortungsbewussten Umgang der Schülerinnen und Schüler mit
persönlichen Daten hingewirkt.

\section{Ergänzende Regeln für die Nutzung außerhalb des Unterrichts zu
unterrichtlichen Zwecken}
\subsection{Nutzungsberechtigung}
Außerhalb des Unterrichts kann in der Nutzungsordnung im Rahmen der
pädagogischen Arbeit ein Nutzungsrecht gewährt werden. Die Entscheidung
hierüber und auch, welche Dienste genutzt werden können, trifft die Schule
unter Beteiligung der schulischen Gremien. Wenn ein solches Nutzungsrecht
geschaffen wird, sind alle Nutzer über die einschlägigen Bestimmungen der
Nutzungsordnung zu unterrichten. Die Schülerinnen und Schüler, sowie im Falle
der Minderjährigkeit ihre Erziehungsberechtigten, versichern durch ihre
Unterschrift, dass sie diese Ordnung anerkennen.
\subsection{Aufsichtspersonen}
Die Schule hat eine weisungsberechtigte Aufsicht sicherzustellen, die im
Aufsichtsplan einzutragen ist. Dazu können neben Lehrkräften und sonstigen
Bediensteten der Schule auch Eltern eingesetzt werden. Charakterlich geeignete
Schülerinnen und Schüler können als Ergänzung bei der Erfüllung der
Aufsichtspflicht eingesetzt werden. Diesbezüglich gilt es jedoch zu beachten,
dass der Einsatz von Eltern, sonstigen Dritten sowie Schülerinnen und Schülern
bei der Beaufsichtigung die Schulleitung und die beteiligten Lehrkräfte nicht
von ihrer Letztverwantwortung für die Beaufsichtigung befreit. Folglich muss
die Tätigkeit der genannten Hilfskräfte in geeigneter Weise überwacht werden.

\section{Zuständigkeiten}
\subsection{Verantwortlichkeit der Schulleitung}
Die Schulleitung ist dafür verantwortlich, eine Nutzungsordnung entsprechend
dem in der jeweiligen Schulordnung vorgesehen Verfahren aufzustellen. Sie hat
den Systembetreuer, den Webmaster, die Lehrkräfte wie auch aufsichtführende
Personen über die Geltung der Nutzungsordnung zu informieren. Insbesondere hat
sie dafür zu sorgen, dass die Nutzungsordnung in den Räumen der Schule, in
denen eine Nutzung des Internets möglich ist, angebracht wird. Folgerichtig ist
die Nutzungsordnung auch an dem Ort, an dem Bekanntmachungen der Schule
üblicherweise erfolgen, anzubringen. Die Schulleitung hat die Einhaltung der
Nutzungsordnung stichprobenartig zu überprüfen. Die Schulleitung ist ferner
dafür verantwortlich, dass bei einer Nutzung des Internets im Unterricht und
außerhalb des Unterrichts zu unterrichtlichen Zwecken eine ausreichende
Aufsicht sichergestellt ist. Sie hat diesbezüglich organisatorische Maßnahmen
zu treffen. Des Weiteren ist die Schulleitung dafür verantwortlich, über den
Einsatz technischer Vorkehrungen zu entscheiden.
Die Schulleitung trägt die Verantwortung für die Schulwebsite.
\subsection{Verantwortlichkeit des Systembetreuers}
Der Systembetreuer hat in Abstimmung mit dem Lehrerkollegium, der Schulleitung
und dem Sachaufwandsträger über die Gestaltung und Nutzung der schulischen
IT-Infrastruktur zu entscheiden und regelt dazu die Details und überprüft die
Umsetzung:

\begin{itemize}
	\item{Nutzung der schulischen IT-Infrastruktur (Zugang mit oder ohne
		individuelle Authentifizierung, klassenbezogener Zugang,
	platzbezogener Zugang),}
	\item{Nutzung persönlicher mobiler Endgeräte und Datenspeicher
		(beispielsweise USB-Sticks) im Schulnetz,}
	\item{Technische Vorkehrungen zur Absicherung des Internetzugangs (wie
		etwa Firewallregeln, Webfilter, Protokollierung).}
\end{itemize}

\subsection{Verantwortlichkeit des Webmasters}
Der Webmaster hat in Abstimmung mit dem Lehrerkollegium, der Schulleitung und
gegebenenfalls weiteren Vertretern der Schulgemeinschaft über die Gestaltung
und den Inhalt des schulischen Webauftritts zu entscheiden. Er regelt dazu die
Details und überprüft die Umsetzung. Zu seinen Aufgaben gehören:

\begin{itemize}
	\item{Auswahl eines geeigneten Webhosters in Abstimmung mit dem
		Sachaufwandsträger,}
	\item{Vergabe von Berechtigungen zur Veröffentlichung auf der
		schulischen Website,}
	\item{Überprüfung der datenschutzrechtlichen Vorgaben, insbesondere bei
		der Veröffentlichung persönlicher Daten und Fotos,}
	\item{Regelmäßige Überprüfung der Inhalte der schulischen Webseiten.}
\end{itemize}

\subsection{Verantwortlichkeit der Lehrkräfte}
Die Lehrkräfte sind für die Beaufsichtigung der Schülerinnen und Schüler bei
der Nutzung der IT-Ausstattung und des Internets im Unterricht und außerhalb
des Unterrichts zu unterrichtlichen Zwecken verantwortlich.
\subsection{Verantwortlichkeit der aufsichtführenden Personen}
Die aufsichtführenden Personen haben auf die Einhaltung der Nutzungsordnungen
durch die Schülerinnen und Schüler hinzuwirken.
\subsection{Verantwortlichkeit der Nutzerinnen und Nutzer}
Die Schülerinnen und Schüler haben das Internet verantwortungsbewusst zu
nutzen. Sie dürfen bei der Nutzung der IT-Ausstattung und des Internets nicht
gegen gesetzliche Vorschriften verstoßen. Sie haben die Regelungen der
Nutzungsordnung einzuhalten.

\section{Schlussvorschriften}
Diese Nutzungsordnung ist Bestandteil der gültigen Hausordnung und tritt am
Tage nach ihrer Bekanntgabe durch Aushang in der Schule in Kraft. Einmal zu
jedem Schuljahresbeginn findet eine Nutzerbelehrung statt, die im Klassenbuch
protokolliert wird. Nutzer, die unbefugt Software von den Arbeitsstationen oder
aus dem Netzwerk kopieren oder verbotene Inhalte nutzen, können strafrechlich
sowie zivilrechtlich belangt werden. Zuwiderhandlungen gegen diese
Nutzungsordnung können neben dem Entzug der Nutzungsberechtigung
schulordnungsrechtliche Maßnahmen zur Folge haben.

\end{document}
